%%
%% This is file `ubcsample.tex',
%% generated with the docstrip utility.
%
% The original source files were:
%
% ubcthesis.dtx  (with options: `ubcsampletex')
%% 
%% This file was generated from the ubcthesis package.
%% --------------------------------------------------------------
%% 
%% Copyright (C) 2001
%% Michael McNeil Forbes
%% mforbes@alum.mit.edu
%% 
%% This file may be distributed and/or modified under the
%% conditions of the LaTeX Project Public License, either version 1.2
%% of this license or (at your option) any later version.
%% The latest version of this license is in
%%    http://www.latex-project.org/lppl.txt
%% and version 1.2 or later is part of all distributions of LaTeX
%% version 1999/12/01 or later.
%% 
%% This program is distributed in the hope that it will be useful,
%% but WITHOUT ANY WARRANTY; without even the implied warranty of
%% MERCHANTABILITY or FITNESS FOR A PARTICULAR PURPOSE.  See the
%% LaTeX Project Public License for more details.
%% 
%% This program consists of the files ubcthesis.dtx, ubcthesis.ins, and
%% the sample figures fig.eps and fig.fig.
%% 
%% This file may be modified and used as a base for your thesis without
%% including the licence agreement as long as the content (i.e. textual
%% body) of the file is completely rewritten. You must, however, change
%% the name of the file.
%% 
%% This file may only be distributed together with a copy of this
%% program. You may, however, distribute this program without generated
%% files such as this one.
%% 

% This Sample thesis requires \LaTeX2e
\NeedsTeXFormat{LaTeX2e}[1995/12/01]
\ProvidesFile{ubcsample.tex}[2010/12/09 v1.68 ^^J
 University of British Columbia Sample Thesis]
% This is the \documentclass[]{} command.  The manditory argument
% specifies the "flavour" of thesis (ubcthesis for UBC).  The
% optional arguments (in []) specify options that affect how the
% thesis is displayed.  Please see the ubcthesis documentation for
% details about the options.
\documentclass[msc,oneside]{ubcthesis}
%
% To compile this sample thesis, issue the following commands:
% latex ubcsample
% bibtex ubcsample
% latex ubcsample
% latex ubcsample
% latex ubcsample
%
% To view use xdvi (on unix systems):
% xdvi ubcsample.dvi
%
% To make a postscript file, use dvips:
% dvips -o ubcsample.ps ubcsample.dvi
%
% To view the postscript file, use ghostview or gv (on unix systems):
% gv ubcsample.ps
%
%************************************************
% Optional packages.
%
% The use of these packages is optional, but they provide various
% tools for more flexible formating.  The sample thesis uses these,
% but if you remove the example code, you should be able to exclude
% these packages.  Only standard packages have been described here;
% they should be installed with any complete LaTeX instalation, but
% if not, you can find them at the Comprehensive TeX Archive Network
% (CTAN): http://www.ctan.org/
%

%******** afterpage ***************************
% This package allows you to issue commands at the end of the current
% page.  A good use for this is to use the command
% \afterpage{\clearpage} right after a figure.  This will cause the
% figure to be inserted on the page following the current one (or on
% the current page if it will fit) but will not break the page in the
% middle.
\usepackage{afterpage}

%******** float *********************************
% This package allows you to customize the style of
% "floats"---floating objects such as figures and tables.  In
% addition, it allows you to define additional floating objects which
% may be included in a list similar to that produces by \listoftables
% and \listoffigures.  Common uses include introducing floats for
% programs and other code bits in Compute Science and Chemical Schema.
\usepackage{float}

%******** tocloft *******************************
% This package allows you to customize and define custom lists such
% as a list of programs or Chemical Scheme.  Note: if you use the
% subfigure package, you must specify that you do as an option here.
% The title option uses the default formatting.  We do not use this
% here as the default formatting is acceptable.  Use the float
% package instead unless you need the extra formatting control
% provided by tocloft.
%\usepackage[subfigure, titles]{tocloft}

%******** alltt *********************************
% The alltt package allows you to include files and have them
% formatted in a verbatim fashion.  This is useful for including
% source code from an additional file.
%\usepackage{alltt}

%******** listings ******************************
% The listings package may be used to include chunks of source code
% and has facilities for pretty-printing many languages.
%\usepackage{listings}

%******** longtable *****************************
% The longtable package allows you to define tables that span
% multiple pages.
\usepackage{longtable}

%******** graphics and graphicx *****************
% This allows you to include encapsulated postscript files.  If you
% don't have this, comment the \includegraphics{} line following the
% comment "%includegraphics" later in this file.
\usepackage{graphicx}

%******** subfigure *****************************
% The subfigure package allows you to include multiple figures and
% captions within a single figure environment.
%\usepackage{subfigure}

%******** here **********************************
% The here package gives you more control over the placement of
% figures and tables.  In particular, you can specify the placement
% "H" which means "Put the figure here" rather than [h] which means
% "I would suggest that you put the figure here if you think it looks
% good."
%\usepackage{here}

%******** pdflscape ********************************
% This allows you to include landscape layout pages by using the
% |landscape| environment.  The use of |pdflscape| is preferred over
% the standard |lscape| package because it automatically rotates the
% page in the pdf file for easier reading.  (Thanks to Joseph Shea
% for pointing this out.)
\usepackage{pdflscape}

%******** natbib ********************************
% This is a very nice package for bibliographies.  It includes options
% for sorting and compressing bibliographic entries.
\usepackage[numbers,sort&compress]{natbib}

%******** psfrag ******************************
% This allows you to replace text in postscript pictures with formated
% latex text.  This allows you to use math in graph labels
% etc. Uncomment the psfrag lines following the "%psfrag" comment
% later in this file if you don't have this package.  The replacements
% will only be visible in the final postscript file: they will be
% listed in the .dvi file but not performed.
\usepackage{psfrag}

%******** hyperref *****************************
% Please read the manual:
% http://www.tug.org/applications/hyperref/manual.html
%
% This adds hyperlinks to your document: with the right viewers (later
% versions of xdvi, acrobat with pdftex, latex2html etc.) this will
% make your equation, figure, citation references etc. hyperlinks so
% that you can click on them.  Also, your table of contents will be
% able to take you to the appropriate sections.  In the viewers that
% support this, the links often appear with an underscore.  This
% underscore will not appear in printed versions.
%
% Note: if you do not use the hypertex option, then the dvips driver
% may be loaded by default.  This will cause the entries in the list
% of figures and list of tables to be on a single line because dvips
% does not deal with hyperlinks on broken lines properly.
%
% NOTE: HYPERREF is sensitive to the ORDER in which it is LOADED.
% For example, it must be loaded AFTER natbib but BEFORE newly
% defined float environments.  See the README file with the hyperref
% for some help with this.  If you have some very obscure errors, try
% first disabling hyperref.  If that fixes the problem, try various
% orderings.
%
% Note also that there is a bug with versions before 2003/11/30
% v6.74m that cause the float package to not function correctly.
% Please ensure you have a current version of this package.  A
% warning will be issued if you leave the date below but do not have
% a current version installed.
%
% Some notes on options: depending on how you build your files, you
% may need to choose the appropriate option (such as [pdftex]) for the
% backend driver (see the hyperref manual for a complete list).  Also,
% the default here is to make links from the page numbers in the table
% of contents and lists of figures etc.  There are other options:
% excluding the [linktocpage] option will make the entire text a
% hyperref, but for some backends will prevent the text from wrapping
% which can look terrible.  There is a [breaklinks=true] option that
% will be set if the backend supports (dvipdfm for example supports
% it but does not work with psfrag.)
%
% Finally, there are many options for choosing the colours of the
% links.  These will be included by default in future versions but
% you should probably consider changing some now for the electronic
% version of your thesis.
\usepackage[unicode=true,
  linktocpage,
  linkbordercolor={0.5 0.5 1},
  citebordercolor={0.5 1 0.5},
  linkcolor=blue]{hyperref}

% If you would like to compile this sample thesis without the
% hyperref package, then you will need to comment out the previous
% \usepackage command and uncomment the following command which will
% put the URL's in a typewriter font but not link them.
%\newcommand\url[1]{\texttt{#1}}

%******** setspace *******************************
% The setspace package allows you to manually set the spacing of the
% file.  UBC may require 1.5 spacing for microfilming of theses.  In
% this case you may obtain this by including this package and issuing
% one of the following commands:
%\usepackage{setspace}
%\singlespacing
%\onehalfspacing
%\doublespacing

% These commands are optional.  The defaults are shown.  You only
% need to include them if you need a different value
\institution{The University Of British Columbia}

% If you are at the Okanagan campus, then you should specify these
% instead.
%\faculty{The College of Graduate Studies}
%\institutionaddress{Okanagan}
\faculty{The Faculty of Electrical and Computer Engineering}
\institutionaddress{Vancouver}

% You can issue as many of these as you have...
\previousdegree{B.ScEng, The University of New Brunswick, 2008}

% You can override the option setting here.
\degreetitle{Masters of Applied Science}

% These commands are required.
\title{A Sample UBC Thesis}
\subtitle{With a Subtitle}
\author{William Joseph Gaudet}
\copyrightyear{2011}
\submitdate{\monthname\ \number\year} % The "\ " is required after
                                      % \monthname to prevent the
                                      % command from eating the space.
\program{Engineering}

% These commands are presently not required for UBC theses as the
% advisor's name and title are not presently required anywhere.
%\advisor{Ariel R.~Zhitnitsky}
%\advisortitle{Professor of Physics}

% One might want to override the format of the section and chapter
% numbers.  This shows you how to do it.  Note that the current
% format is acceptable for submission to the FoGS: If you wish to modify
% these, you should check with the FoGS explicity. prior to making
% the modifications.
\renewcommand\thepart         {\Roman{part}}
\renewcommand\thechapter      {\arabic{chapter}}
\renewcommand\thesection      {\thechapter.\arabic{section}}
\renewcommand\thesubsection   {\thesection.\arabic{subsection}}
\renewcommand\thesubsubsection{\thesubsection.\arabic{subsubsection}}
\renewcommand\theparagraph    {\thesubsubsection.\arabic{paragraph}}
\renewcommand\thesubparagraph {\theparagraph.\arabic{subparagraph}}

\setcounter{tocdepth}{2}
\setcounter{secnumdepth}{2}

% Here is an example of a "Program" environment defined with the
% "float" package.  The list of programs will be stored in the file
% ubcsample.lop and the numbering will start with the chapter
% number.  The style will be "ruled".
\floatstyle{ruled}
\newfloat{Program}{htbp}{lop}[chapter]

% Here is the start of the document.
\begin{document}

%% This starts numbering in Roman numerals as required for the thesis
%% style and is mandatory.
\frontmatter

%%% The order of the following components should be preserved.  The order
%%% listed here is the order currently required by FoGS:        \\
%%% Title (Mandatory)                                           \\
%%% Preface (Manditory if any collaborator contributions)       \\
%%% Abstract (Mandatory)                                        \\
%%% List of Contents, Tables, Figures, etc. (As appropriate)    \\
%%% Acknowledgements (Optional)                                 \\
%%% Dedication (Optional)                                       \\

\maketitle                      %% Mandatory
\begin{abstract}                %% Mandatory -  maximum 350 words
Today the average north american carries a computer several orders of magnitude more powerful than the computer used to guide the Apollo space craft to the moon and safely back. However, with this dramatic increase in both ubiquity and power of personal mobile computing, has come a host of applications that require significantly more computational power than is available in today's smart phones. Additionally, battery life of mobile devices is still a significantly limiting factor when doing large quantities of computation on mobile platforms.

With these factors in mind this research investigates the gains in performance, quality of computation, and battery life which can be made possible by real time offloading of computation from a mobile device to a server platform.

Furthermore, it details the design and implementation of a framework for offloading computation from mobile devices which has at its core the goal of limiting the cognitive overhead for the developer to conduct such an offload.

\end{abstract}

\chapter{Preface} % Manditory if any of the conditions are met

You must include a preface if any part of your research was partly or
wholly published in articles, was part of a collaboration, or required
the approval of UBC Research Ethics Boards.

The Preface must include the following:

\begin{itemize}
\item A statement indicating the relative contributions of all
  collaborators and co-authors of publications (if any), emphasizing
  details of your contribution, and stating the proportion of research
  and writing conducted by you.
\item A list of any publications arising from work presented in the
  dissertation, and the chapter(s) in which the work is located.
\item The name of the particular UBC Research Ethics Board, and the
  Certificate Number(s) of the Ethics Certificate(s) obtained, if
  ethics approval was required for the research.
\end{itemize}

%%% Sections and subsections etc. in the Preface should in general
%%% not be listed in the table of contents, so use the starred form
%%% of \section etc.
\section*{Examples}
Chapter~\ref{cha:apple_ref} is based on work conducted in UBC's Maple
Syrup Laboratory by Dr. A.  Apple, Professor B. Boat, and Michael
McNeil Forbes. I was responsible for tapping the trees in forests X
and Z, conducted and supervised all boiling operations, and performed
frequent quality control tests on the product.

A version of chapter~\ref{cha:apple_ref} has been
published~\cite{Apple:2010}. I conducted all the testing and wrote
most of the manuscript. The section on ``Testing Implements'' was
originally drafted by Boat, B.  Check the first pages of this
chapter to see footnotes with similar information.

Note that this preface must come before the table of contents.  Note
also that this section ``Examples'' should not be listed in the table
of contents, so we have used the starred form: \verb|\section*{Example}|.

\tableofcontents                %% Mandatory
\listoftables                   %% Mandatory if thesis has tables
\listoffigures                  %% Mandatory if thesis has figures
\listof{Program}{List of Programs} %% Optional
%% Any other lists should come here, i.e.
%% Abbreviation schemes, definitions, lists of formulae, list of
%% schemes, glossary, list of symbols etc.

\chapter{Acknowledgements}      %% Optional

\chapter{Dedication} %% Optional


% Any other unusual prefactory material should come here before the
% main body.

% Now regular page numbering begins.
\mainmatter

% Parts are the largest structural units, but are optional.
%\part{Thesis}


%% This changes the headings and chapter titles (no numbers for
%% example).
\backmatter

%% This file is setup to use a bibtex file sample.bib and uses the
%% plain style.  Other styles may be used depending on the conventions
%% of your field of study.
%%
%%% Note: the bibliography must come before the appendices.
\bibliographystyle{plain}
\bibliography{sample}

%% Use this to reset the appendix counter.  Note that the FoGS
%% requires that the word ``Appendices'' appear in the table of
%% contents either before each appendix lable or as a division
%% denoting the start of the appendices.  We take the latter option
%% here.  This is ensured by making the \texttt{appendicestoc} option
%% a default option to the UBC thesis class.

%%% If you only have one appendix, please uncomment the following line.
% \renewcommand{\appendicesname}{Appendix}
\appendix
\chapter{First Appendix}
Here you can have your appendices.  Note that if you only have a
single appendix, you should issue
\verb|\renewcommand{\appendicesname}{Appendix}| before calling
\verb|\appendix| to display the singular ``Appendix'' rather than the
default plural ``Appendices''.

\chapter{Second Appendix}
Here is the second appendix.

%% Indices come here if you have them.

\chapter*{Additional Information}
This chapter shows you how to include additional information in your
thesis, the removal of which will not affect the submission.  Such
material should be removed before the thesis is actually submitted.

First, the chapter is unnumbered and not included in the Table of
Contents.  Second, it is the last section of the thesis, so its
removal will not alter any of the page numbering etc. for the previous
sections.  Do not include any floats, however, as these will appear in
the initial lists.

The \texttt{ubcthesis} \LaTeX{} class has been designed to aid you in
producing a thesis that conforms to the requirements of The
University of British Columbia Faculty of Graduate Studies (FoGS).

Proper use of this class and sample is highly recommended---and should
produce a well formatted document that meets the FoGS requirement.
Notwithstanding, complex theses may require additional formatting that
may conflict with some of the requirements.  We therefore \emph{highly
  recommend} that you consult one of the FoGS staff for assistance and
an assessment of potential problems \emph{before} starting final
draft.

While we have attemped to address most of the thesis formatting
requirements in these files, they do not constitute an official set of
thesis requirements.  The official requirements are available at the
following section of the FoGS web site:
\begin{center}
  \begin{tabular}{|l|}
    \hline
    \url{http://www.grad.ubc.ca/current-students/dissertation-thesis-preparation}\\
    \hline
  \end{tabular}
\end{center}
We recommend that you review these instructions carefully.

\end{document}
\endinput
%%
%% End of file `ubcsample.tex'.
